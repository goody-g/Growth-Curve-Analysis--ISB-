\documentclass[11pt]{article}
 
\usepackage[small,compact]{titlesec}
\usepackage[margin=1in,nohead]{geometry}
\setlength\parskip{0.1in}
\setlength\parindent{0in}
 
\usepackage[pdftex]{graphicx}
\usepackage{mdwlist}
\usepackage{amsmath}
\usepackage{amsthm}
\usepackage{amsfonts}
\usepackage{setspace}
%\usepackage{dingbat}
 
 %\usepackage[T1]{fontenc}
 %\usepackage[light,math]{iwona}

\usepackage{times}

\newenvironment{example}
{\ttfamily \footnotesize >}
{\normalsize \rmfamily}

\newcommand{\inlex}[1]{\ttfamily\small{#1}\rmfamily}

%\newenvironment{columns}{
%start matter
%\begin{list}{}{} %label code, %body code
%}{
%end matter
%}




\begin{document}
 
\begin{center} \LARGE{Explanation of Information and Parameters for GCAF}\small{ - Goodwin Gibbins} \end{center}

%\begin{spacing}{.1}
 
\section{Meta-Information Sections}
These sections make up the columns in the database information files and the \inlex{\$Info} portion of class \inlex{gAnalysis} objects.


\begin{description}
%\setlength{\labelwidth}{30pt}
%\setlength{\itemindent}{5pt}

\item[Initials] The initials of the experimenter. Part of the unique identification of a growth curve.
\begin{list}{-}{}
\item FYL - Fang Yin Lo
\item ST - Serdar Turkarslan
\item LP - Lee Pang
\item	KB - Karlyn Beer
\end{list}

\item[Date] The date the experiment was carried out or a consecutive day if multiple experiments occured at the same time. Part of the unique identification of a growth curve.
\begin{list}{-}{}
\item YYYYMMDD - for example, April 15, 2010 would be 20100415
\end{list}

\item[Well.Number] The well number for a bioscreen experiment, matching the well number on the results file and extracted from the labels file. Part of the unique identification of a growth curve.

\item[Well.Name] The label-file entry for a given well number, to be translated to fill in columns of information file.

\item[Media] A description of the media use.
\begin{list}{-}{}
\item CM - complete media
\item CDM - 
\item NA - no media in well
\item More...
\end{list}


\item[Background] The strain of bacteria used for the experiment.
\begin{list}{-}{}
\item NRC-1 - wild type
\item ura3 - d-ura3, means bacteria were grown with uracil
\item NA - no bacteria in well
\end{list}

\item[Knockout] A gene or list of genes removed. Multiple knockouts separated by \& (need to develop search method for this). Naming is based on the ORF Name from the annotated genome search on the Baliga lab website.
\begin{list}{-}{}
\item more....
\item NA - no knockout
\end{list}

\item[Overexpression] A gene or list of genes overexpressed. Multiple knockouts separated by \& (need to develop search method for this). Naming is based on the ORF Name from the annotated genome search on the Baliga lab website.
\begin{list}{-}{}
\item more....
\item NA - no overexpression
\end{list}

\item[Biological.Replicate] The number of a biological replicate
\begin{list}{-}{}
\item NA - not listed or only one biological replicate.
\end{list}

\item[Temp] The temperature the experiment is run at.
\begin{list}{-}{}
\item NA - unspecified - probably 37$^\circ$.
\end{list}

\item[pH] The pH of the media.
\begin{list}{-}{}
\item NA - unspecified/unperterbed.
\end{list}

\item[Culture.OD] the OD of the culture before dilution for growth curve run.
\begin{list}{-}{}
\item NA - not recorded
\end{list}

\item[NaCl.Concentration] The molarity of salt in the media. 
\begin{list}{-}{}
\item NA - XX M
\end{list}

\item[XX.Concentration] The Cu$^{2+}$ (Cu2p), Fe, Mn, Ni, Co, or Zn concentration in mM.
\begin{list}{-}{}
\item NA - no metal.
\end{list}

%http://baliga.systemsbiology.net/drupal/content/halobacterium-nrc1
\end{description}


\section{Parameters}
The parameters are based on mathematical algorithms which can be found in the \inlex{gSplineFit.r} or \inlex{gFitXXX.r} functions. 

\begin{description}
\item[A] - the maximum growth (often just the final recorded growth since maximum not acheived.)
\item[time.A] - the time which A occurs at.
\item[mu] - the maximum growth rate (sometimes of first hump, sometimes second. Based on derivative of spline fit)
\item[time.mu] - the time which the maximum growth rate occurs.
\item[y.mu] - the cell density at the maximum growth rate.
\item[lambda] - the lag time - the intercept of the maximum growth rate with the spline initial growth level. (Subject to errors capturing the wrong maximum growth rate).
\item[integral] - the area under the growth curve.
\item[initial.od] - the initial cell density, extrapolated from the spline fit to avoid noise.
\item[time.max] - the length of the experiment.
\item[trajectory] - the slope at the end of the curve (more positive would mean that the actual A was much higher than the measured A)
\item[max.secderiv] - the maximum of the spline-interpolated second derivative.
\item[max.secderiv.time] - time at which the maximum pf the second derivative occurs.
\item[max.secderiv.index] - index location of the maximum of the second derivative.
\end{description}


%\end{spacing}
\end{document}
 
 